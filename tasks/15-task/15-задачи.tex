\documentclass[14pt]{extreport}
\usepackage[utf8]{inputenc}
\usepackage[russian]{babel}
\usepackage{indentfirst}
\usepackage{amsmath}
\usepackage{vmargin}
\usepackage{minted}
\usepackage{graphicx}
\graphicspath{ {../../data/15-images/} }
\usepackage{hyperref}
\hypersetup{
	colorlinks,
	citecolor=black,
	filecolor=black,
	linkcolor=black,
	urlcolor=blue
}
\setmarginsrb{2cm}{2cm}{1.5cm}{2cm}{0pt}{0mm}{0pt}{13mm}
\usepackage{xcolor}
\definecolor{LightGray}{gray}{0.97}
\sloppy
\title{Решение задач для 15-го задания}
\date{}

\begin{document}
\maketitle
\tableofcontents

\chapter{Задачки на делимость}
\section{Задача на делимость (первый способ)}
\begin{figure}[h]
	\includegraphics[width=\textwidth]{121-polyakov}
\end{figure}

В данном случае мы воспользуемся флажком, который будет сигнализировать нам о том, подходит ли данное число или нет.

\begin{minted}[bgcolor=LightGray]{python}
def Del(n, m):
    return n % m == 0


def func(x, a):
    return ((not Del(x, a)) and Del(x, 21)) <= (not Del(x, 14))


for a in range(10_000, 1, -1):
    a_correct = True
    for x in range(1, 10_000):
        if func(x, a) is not True:
            a_correct = False
            break

    if a_correct:
        print(a)
        break
\end{minted}

Нам нужно наибольшее число \texttt{a}, поэтому мы идем по убыванию.

С помощью переменной \texttt{a\_correct} я обозначаю условие, при котором все $x$ удовлетворяют формуле.

Если хотя бы один $x$ не удовлетворяет формуле, то переводим флажок в статус \texttt{False}.

Как только мы находим нужное значение \texttt{a}, выводим его и выходим.

Ответ: \boxed{42}

\newpage
\section{Задача на делимость (второй способ)}
\begin{figure}[h]
    \includegraphics[width=\textwidth]{128-polyakov}
\end{figure}

Если в предыдущем примере мы использовали флажок, то теперь воспользуемся особенностями Python, т.е. конструкцией \texttt{for ... else}.

\begin{minted}[bgcolor=LightGray]{python}
def Del(n, m):
    return n % m == 0


def func(x, a):
    return (not Del(x, 18)) <= ((not Del(x, a)) <= (not Del(x, 12)))


for a in range(10_000, 1, -1):
    for x in range(1, 10_000):
        if func(x, a) is not True:
            break
    else:
        print(a)
        break
\end{minted}

Логика конструкции \texttt{for ... else} заключается в следующем: если цикл \texttt{for} не завершился аварийно (с помощью \texttt{break}, например), то в случае успеха мы переходим в блок \texttt{else}.

Ответ: $\boxed{12}$

\textit{Нужно стараться перебирать значения $x$ побольше}.

\newpage
\section{Задача на делимость (третий способ)}
\begin{figure}[h]
    \includegraphics[width=\textwidth]{131-polyakov.png}
\end{figure}

Теперь мы воспользуемся генераторным выражением.

\begin{minted}[bgcolor=LightGray,breaklines=true]{python}
def Del(n, m):
    return n % m == 0


def func(x):
    return (Del(x, a) and Del(x, 12)) <= (Del(x, 42) or (not Del(x, 12)))

for a in range(1, 10_000):
    results = [func(x) for x in range(1, 10_000)]

    if all(results):
        print(a)
        break
\end{minted}

Как это вообще работает, если мы убрали \texttt{a}? Дело в том, что здесь $a$ выступает в роли \href{https://docs.python.org/3/faq/programming.html#what-are-the-rules-for-local-and-global-variables-in-python}{глобальный (\textit{global}) переменной}. В данном случае, когда мы входим в функцию \texttt{func}, то Python ищет нашу переменную \texttt{a} внутри функции, в которой она используется. Если не удалось ее найти, Python ищет ее в глобальной области видимости. И тут уже она находится.

С помощью \href{https://peps.python.org/pep-0289/}{генераторного выражения} мы создаем список с результатами значений формул.

Благодаря \texttt{all} проверяем, при всех ли значениях \texttt{x} формула истинна.

Ответ: $\boxed{7}$
\end{document}
